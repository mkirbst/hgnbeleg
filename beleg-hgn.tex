%Wir verwenden eine DIN-A4-Seite und die Schriftgröße 12.
\documentclass[a4paper,12pt]{scrartcl} 
%%Basiert auf folgender Quelle: <!-- m --><a class="postlink" href="http://www.studieren-info.de/downloads/vorlage-hausarbeit.tex">http://www.studieren-info.de/downloads/ ... arbeit.tex</a><!-- m -->
 
%Diese drei Pakete benötigen wir für die Umlaute, Deutsche Silbentrennung etc.
%Apple-Nutzer sollten anstelle von \usepackage[latin1]{inputenc} das Paket \usepackage[applemac]{inputenc} verwenden
%% \usepackage[latin1]{inputenc}
%%apt-get install texlive-lang-german damit ngerman keine Probleme mehr macht !!
\usepackage[utf8]{inputenc} 
\usepackage[T1]{fontenc}
\usepackage[ngerman]{babel}
 
%Das Paket erzeugt ein anklickbares Verzeichnis in der PDF-Datei.
\usepackage{hyperref}
 
%Das Paket wird für die anderthalb-zeiligen Zeilenabstand benötigt
\usepackage{setspace}
 
%mathematische Zeichen/Symbole
\usepackage{amsmath}
\usepackage{amssymb}

%Einrückung eines neuen Absatzes
\setlength{\parindent}{0em}
 
%Definition der Ränder
\usepackage[paper=a4paper,left=30mm,right=30mm,top=30mm,bottom=30mm]{geometry} 
 
%Abstand der Fussnoten
\deffootnote{1em}{1em}{\textsuperscript{\thefootnotemark\ }}
 
%Regeln, bis zu welcher Tiefe (section,subsection,subsubsection) Überschriften angezeigt werden sollen (Anzeige der Überschriften im Verzeichnis / Anzeige der Nummerierung)
\setcounter{tocdepth}{3}
\setcounter{secnumdepth}{3}
 
\begin{document}
 
%Beginn der Titelseite
\begin{titlepage}
\begin{small}
\vfill {HTWK Leipzig\\
Fachbereich IMN \\
Wintersemester 2013/2014}
\end{small}
 
\begin{center}
\begin{Large}
\vfill {\textsf{\textbf{
Internet Protokoll Version 6\\
}}}
\end{Large}
Beleg im Fach Hochgeschwindigkeitsnetze
\end{center}
 
\begin{small}
\vfill Marcel Kirbst \\ Sieglitz 39 \\ 06618 Molauer Land \\
marcel.kirbst@htwk-leipzig.de\\
\today
\end{small}
 
\end{titlepage}
%Ende der Titelseite
 
%Inhaltsverzeichnis (aktualisiert sich erst nach dem zweiten Setzen)
\tableofcontents
\thispagestyle{empty}
 
%Beginn einer neuen Seite
\clearpage
 
%Anderthalbzeiliger Zeilenabstand ab hier
\onehalfspacing
 
\pagestyle{plain}
 
\section{Einleitung}
Dieser Beleg befasst sich mit der Vorstellung der Protokollfamilie IPv6. Neben einem kompakten Überblick über IPv6 soll auf die Notwendigkeit sowie die Vorzüge von IPv6 im Vergleich zu IPv4, der derzeit weltweit am meisten im Internet genutzten Protokollfamilie,  eingegangen werden.
 
\section{Warum IPv6 ?}
Das Internet ist heute überall im täglichen Leben präsent und wird von der UN inzwischen zu den Menschenrechten gezählt.\cite{uninet} Die im Internet heute noch am häufigsten verwendete Protokollfamilie ist IPv4. Wie der folgende Abschnitt darlegt, ist die weitere Verwendung von IPv4 aber mit immer mehr Problemen behaftet.

\subsection{Probleme mit IPv4}
Zu Beginn der Entwicklung des Internets war nicht abzusehen wie rasant und erfolgreich sich diese vollziehen würde. Ziel war einige wenige Rechnersysteme mit einander zu verbinden. Daher schien es beispielsweise mehr als ausreichend einen 32 Bit breiten Adressraum zu wählen, der somit $2^{32} = 4.294.967.296$ eindeutige Netzwerkadressen zulässt. In den 1970er Jahren entsprach das immerhin annährend einer IPv4-Adresse pro Erdenbewohner. 
 
\subsection{Zwischenüberschrift}
Hier beginnt der zweite Unterabschnitt des ersten Hauptteils.
 
\section{zweiter Hauptteil}
Hier beginnnt der zweite Hauptteil des Belegs.
 
\subsection{Zwischenüberschrift}
Hier beginnt der erste Unterabschnitt des zweiten Hauptteils.
 
\subsection{Zwischenüberschrift}
Hier beginnt der zweite Unterabschnitt des zweiten Hauptteils.
 
\section{Schluss}
Dies ist der Schlussteil.
%Beginn einer neuen Seite
\clearpage
 
\section{Literaturverzeichnis}

\begin{thebibliography}{50}

\bibitem{uninet}    \url{https://www.un.org/Depts/german/menschenrechte/a-hrc-20-L.13.pdf}

\end{thebibliography}
 
 
\end{document}
%-------------------
%Hier endet der Text deiner Hausarbeit
%-------------------